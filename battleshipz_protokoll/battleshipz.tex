\documentclass[paper=a4, fontsize=11pt]{scrartcl} % A4 paper and 11pt font size
\usepackage[utf8]{inputenc}

\usepackage[T1]{fontenc} % Use 8-bit encoding that has 256 glyphs
\usepackage[ngerman]{babel} % English language/hyphenation
\usepackage{amsmath,amsfonts,amsthm} % Math packages

\usepackage{graphicx}
\usepackage{float}

\usepackage{sectsty} % Allows customizing section commands
\allsectionsfont{\centering \normalfont\scshape} % Make all sections centered, the default font and small caps

\usepackage{fancyhdr} % Custom headers and footers
\pagestyle{fancyplain} % Makes all pages in the document conform to the custom headers and footers
\fancyhead{} % No page header - if you want one, create it in the same way as the footers below
\fancyfoot[L]{} % Empty left footer
\fancyfoot[C]{} % Empty center footer
\fancyfoot[R]{\thepage} % Page numbering for right footer
\renewcommand{\headrulewidth}{0pt} % Remove header underlines
\renewcommand{\footrulewidth}{0pt} % Remove footer underlines
\setlength{\headheight}{13.6pt} % Customize the height of the header

\numberwithin{equation}{section} % Number equations within sections (i.e. 1.1, 1.2, 2.1, 2.2 instead of 1, 2, 3, 4)
\numberwithin{figure}{section} % Number figures within sections (i.e. 1.1, 1.2, 2.1, 2.2 instead of 1, 2, 3, 4)
\numberwithin{table}{section} % Number tables within sections (i.e. 1.1, 1.2, 2.1, 2.2 instead of 1, 2, 3, 4)

\setlength\parindent{0pt} % Removes all indentation from paragraphs - comment this line for an assignment with lots of text

\newcommand {\horrule}[1]{\rule{\linewidth}{#1}} % Create horizontal rule command with 1 argument of height

\title {
  \normalfont \normalsize
  \textsc{HAW Hamburg} \\ [25pt] % Your university, school and/or department name(s)
  \textsc{Technik und Technologie von vernetzen Systeme} \\ [15pt]
  \horrule{0.5pt} \\[0.4cm] % Thin top horizontal rule
  \huge Distributed Battleships \\ [15pt] % The assignment title
  \small  Prof. Dr. T. Schmidt \\
  \horrule{2pt} \\[0.5cm] % Thick bottom horizontal rule
}

\author{Fabien Lapok, Matthias Nitsche}

\date{\normalsize\today}

\begin{document}

\maketitle

\section{Spielablauf}

Verteiltes Schiffe versenken wird auf Basis von Chord - einem strukturieten Peer-to-Peer System - realisiert. Im Hintergrund wird ein Ring auf basis einer verteilten Hashtabelle aufgebaut. Peers haben nur begrenzte sicht über andere Peers, außer den direkten Vorgängern. Andere Peers werden erst über Zeit aufgedeckt. Es obliegt den jeweiligen Teams auszuwerten und fest zu halten wie sich die anderen Spieler auf der Hashtabelle anordnen und welche Schiffe schon getroffen wurden. Hierfür wird ein Broadcast Mechanismus verwendet der über Chord an alle Teilnehmer im Netzwerk propagiert werden muss.

\section{Strategie}

Wir haben 2 Spielstrategien, die eine ist uniform (zufällig) auf einen anderen Spieler zu schießen, die andere auf den zu schießen der die minimalsten Schiffe übrig hat. Die Einschränkung ist das wir nicht auf uns schießen und keine Felder anschießen die schon beschossen wurden.

Da wir die anderen Spieler am Anfang nicht kennen, beschießen wir zunächst immer zufällig einen. Über Zeit gewinnen wir neue Informationen und können so auf die mit den meisten beschossenen Feldern schießen.


\end{document}
