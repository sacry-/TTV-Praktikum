\documentclass[paper=a4, fontsize=11pt]{scrartcl} % A4 paper and 11pt font size
\usepackage[utf8]{inputenc}

\usepackage[T1]{fontenc} % Use 8-bit encoding that has 256 glyphs
\usepackage[ngerman]{babel} % English language/hyphenation
\usepackage{amsmath,amsfonts,amsthm} % Math packages

\usepackage{graphicx}
\usepackage{float}

\usepackage{sectsty} % Allows customizing section commands
\allsectionsfont{\centering \normalfont\scshape} % Make all sections centered, the default font and small caps

\usepackage{fancyhdr} % Custom headers and footers
\pagestyle{fancyplain} % Makes all pages in the document conform to the custom headers and footers
\fancyhead{} % No page header - if you want one, create it in the same way as the footers below
\fancyfoot[L]{} % Empty left footer
\fancyfoot[C]{} % Empty center footer
\fancyfoot[R]{\thepage} % Page numbering for right footer
\renewcommand{\headrulewidth}{0pt} % Remove header underlines
\renewcommand{\footrulewidth}{0pt} % Remove footer underlines
\setlength{\headheight}{13.6pt} % Customize the height of the header

\numberwithin{equation}{section} % Number equations within sections (i.e. 1.1, 1.2, 2.1, 2.2 instead of 1, 2, 3, 4)
\numberwithin{figure}{section} % Number figures within sections (i.e. 1.1, 1.2, 2.1, 2.2 instead of 1, 2, 3, 4)
\numberwithin{table}{section} % Number tables within sections (i.e. 1.1, 1.2, 2.1, 2.2 instead of 1, 2, 3, 4)

\setlength\parindent{0pt} % Removes all indentation from paragraphs - comment this line for an assignment with lots of text

\newcommand {\horrule}[1]{\rule{\linewidth}{#1}} % Create horizontal rule command with 1 argument of height

\title {
  \normalfont \normalsize
  \textsc{HAW Hamburg} \\ [25pt] % Your university, school and/or department name(s)
  \textsc{Technik und Technologie von vernetzen Systeme} \\ [15pt]
  \horrule{0.5pt} \\[0.4cm] % Thin top horizontal rule
  \huge LoWPAN Networking im IoT \\ [15pt] % The assignment title
  \small Prof. Dr. Fohl \\
  \horrule{2pt} \\[0.5cm] % Thick bottom horizontal rule
}

\author{Fabien Lapok, Matthias Nitsche}

\date{\normalsize\today}

\begin{document}

\maketitle

\section{Aufgabe - Laboreinbindung von Gateways und Sensorknoten}

Nach dem Aufsetzen von Szenario 1 haben wir mit Wireshark - siehe Abbildung \ref{fig:hrwireshark} - die Neighbor Discovery von Host zu Router mitgeschnitten.

\begin{figure}[H]
  \centering
  \includegraphics[width=\linewidth]{imgs/host-to-router-interaction-wireshark.png}
  \caption{Host zu Router Interaktion Wireshark Mitschnitt}
  \label{fig:hrwireshark}
\end{figure}

Der Ablauf des Protokolls ist detailliert in RFC 6775 ``''Neighbor Discovery Optimization``'' unter Host-to-Router interaction beschrieben. In Abbildung \ref{fig:hrfluss} ist ein Flussdiagramm was den Austausch vom Raspberry Pi (RPI) als Rouer zum Sensor als Host beschreibt.
\\
\\
Die ersten 4 Schritte laufen Ordnungsgemäß wie in RFC 6775 ab. Sensor fragt nach Router ``Router Solicitation'', RPI antwortet ``Router Advertisment'', Sensor fragt Router nach der Nachbarschaft ``Neighbor Solicitation'' und RPI antwortet mit ``Neighbor Advertisement''. Nach 5 Sekunden wird die Nachbarschaftsanfrage erneut gesendet.

\begin{figure}[H]
  \centering
  \includegraphics[width=0.7\linewidth,height=0.7\columnwidth]{imgs/host-to-router-interaction.png}
  \caption{Host zu Router Interaktion Flussdiagramm}
  \label{fig:hrfluss}
\end{figure}

Uns sind keine Auffälligkeiten im Vergleich zum RFC 6775 aufgefallen.

\section{Aufgabe - RPL Routing im Labornetz}

(Welche Nachrichten und Informationen werden zwischen den RPL-Knoten ausgetauscht.)

Nach dem Aufbau des zweiten Szenarios wurde mithilfe eines Wiresharkmitschnittes der Informationsaustausch zwischen den RPL-Knoten analysiert. Folgende Pakettypen haben sich rauskristalisiert:
\begin{enumerate}
  \item DODAG Information Solicitation (DIS)
  Das DIS Objekt wird beispielsweise initial von einem Knoten verwendet um einem DODAG beizutreten. Analog zur Neighbor Solicitation (6775) kann das DIS verwendet werden um die umliegenden DODAG Knoten zu identifizieren.
  \item DODAG Information Object (DIO)
  Das DIO hält Informationen die es dem Knoten ermöglicht RPL-Infrastruktur aufzudecken, Konfigurationsparameter zu erhalten und sein DODAG-Parent zu setzen.
  \item DODAG advertisement Object (DAO)
  Das DAO wird vom Kindknoten verwendet um Zielinformationen im Aufwärts im DODAG zu verteilen. Beispielsweise als request zum Beitreten des DODAGs.
  \item DAO-ACK
  Der Empfänger eines DAO sended dem Sender ein DAO-ACK als bestätigung.
\end{enumerate}

Mithilfe der durch die Nachrichten bekanntgegebenen Struktur kann ein Graph aufgebaut werden, der es ermöglicht Nachrichten im Netz zu verteilen. In der Regel hat jeder Knoten einen Parent-Knoten, an den er seine Nachrichten weiterleitet. Ein Parent weiß dabei nichts von seinen Kindern, bis diese sich bei ihm melden. Der Parent-Knoten erfährt außerdem aus der DAO-Nachricht des Kind-Knoten, welche weiteren Knoten über das Kind erreichbar sind.

\section{Aufgabe - Datenverteilung und Messung}

Im folgenden machen wir mittels COAP, HTTP ähnliche Anfragen um die REST Ressourcen auf den (RIOT) Sensoren auszulesen. Wir haben die ``led'' mittels ``PUT 1'' und ``PUT 0'' an verschiedenen Rechnern an und ausgeschaltet. Wie zuvor wurde die Kommunikation über Wireshark aufgezeichnet.

\subsection{Sendedauer}

Wir saßen in Reihe 3 und haben den RPL Root, Reihe 2, Reihe 3 (uns) und Reihe 4 mittels coap ``put /led 0|1'' angesteuert. Wir haben die einzelnen Sensoren im Durchschnitt 20mal abgerufen. Dabei ist wie in \ref{fig:sendedauer} zu sehen das je weiter wir uns vom Root entfernen die durchschnittliche Sendedauer höher wird. Der Root hat im Durchschnitt in unter 10 Millisekunden geantwortet, während die restlichen Reihen untereinander nur im Nanosekundenbereich zu unterscheiden sind.

\begin{figure}[H]
  \centering
  \includegraphics[width=0.7\linewidth]{imgs/put-led-response-time.png}
  \caption{Mittlere Sendedauer in Millisekunden}
  \label{fig:sendedauer}
\end{figure}

Wir haben des weiteren geplant den Paketverlust eines pings zu den einzelnen Reihen zu messen. Leider wurde der Mittschnitt nicht korrekt aufgezeichnet und wir konnten die Daten nicht verwerten. Ansonsten ist hier die Hypothese, dass je weiter die Reihen vom Root entfernt sind, der Paketverlust ansteigt. Von Reihe 3 zu Reihe 4 also ein höherer Verlust erwartet wird.

\end{document}
